%!TEX root = ../main.tex

\chapter{Background}
\label{ch:background}

Nowadays, the complexity of software systems is increasing. Even in safety-critical or real-time systems, we can reach a complexity level where we need computer aided analysis techniques to ensure correctness. With the introduction of Internet of Things (\iot), and new mobile communication technologies, systems are becoming mixed-critical. Such systems contain some components that are not checked for safety\,--because it's not possible, or it was not a design aspect --\,, but we need to make sure they.

\section{Embedded systems}

An embedded system is a microprocessor-based system that is built to control a function or range of functions and is not designed to be programmed by the end user in the same way that a \textls{PC} is~\cite{heath2002embedded}.

Often embedded systems consist of more specialized units, which are interconnected by some network e.g.\ \textls{CAN} bus in an automobile. The number of different microcontrollers in a modern car scales from 10 up to even 100 \citep{VEHICLE_DYN}.

Am embeddable microcontroller can be described as a
\begin{itemize}
	\item cost effective
	\item small footprint
	\item low end performance
	\item small program and data memory
\end{itemize}
device. These devices often have specific architectures for best performance, cost, and energy consumption needs.
\needspace{5em}
\noindent On the other hand, a safety-critical system is a
\begin{itemize}
	\item reliable
	\item deterministic
	\item real-time
\end{itemize}
system. Such systems are often called mission-critical. A mission critical system is which failure can cause severe financial loss, natural disaster, or loss of human lives (\cref{sec:mission_critical_system}).

Conflicts in microcontroller and safety-critical specification are serious issues, and must be solved by special techniques which can bridge between the differences.

As \emph{Single Board Computers} (\textls{SBC}) such as Raspberry Pi, BeagleBone, ODROID becoming more and more popular, anybody can easily access embeddable hardware platforms. With the trending of IoT technologies e.g.\ sensor networks, manufacturers are moving towards open source, accessible platform which everybody can use.

\section{Model driven software developement}

Using models to design complex systems is essential in traditional engineering disciplines. Models help us to understand a complex problem and its potential solutions through abstraction. Therefore, it seems obvious that software systems --which are often complex engineering systems-- can benefit greatly from using models and modeling techniques. \citep{pastor2008model}

By using models, the developers are using a common description of the system they are developing. Model based development is highly graphical, and a well designed image of class structure can be self documenting. Apart from documenting, models are interchangeable, and the interchange format is standardized e.g. XMI~\cite{XMI}. In safety-critical systems, model driven software development is highly recommended.

Model based software development consists two main components:
\begin{itemize}
	\item \emph{Tools:} Tools allow the developers to create models, e.g. UML, EMF, SysML
	\item \emph{Process:} The development process describes how to use the tool e.g. SYSMOD, RUP, Y model
\end{itemize}

\begin{figure}
	\centering
	\begin{tikzpicture}[
		start chain=going below,
		squarednode/.style={
			rectangle,
			draw=black,
			very thick,
			minimum height=1.5cm,
			text width=2.5cm,
			inner sep=2mm,
			align=center
		},
		every loop/.style={
			looseness=2,
			text width=2cm
		}
		]
		\node[squarednode] (D1) {Requirements};
		\node[squarednode] (D2) [below = of D1, xshift = 1cm] {Architecture design};
		\node[squarednode] (D3) [below = of D2, xshift = 1cm] {Component design};
		\node[squarednode] (I)  [below = of D3, xshift = 2.5cm] {Implementation};
		\node[squarednode] (V3) [above = of I , xshift = 2.5cm] {System V\&V};
		\node[squarednode] (V2) [above = of V3, xshift = 1cm] {Architecture V\&V};
		\node[squarednode] (V1) [above = of V2, xshift = 1cm] {Unit V\&V};

		\draw[thick,->] (D1) -- (D2) node [midway, left, align=right] {Refine};
		\draw[thick,->] (D2) -- (D3) node [midway, left, align=right] {Refine};
		\draw[thick,->] (D3) -- ( I) node [midway, right, align=left] {};
		\draw[thick,<-] ( I) -- (V3) node [midway, right, align=left] {};
		\draw[thick,<-] (V3) -- (V2) node [midway, right, align=left] {Use};
		\draw[thick,<-] (V2) -- (V1) node [midway, right, align=left] {Use};

		\draw[thick,->] ($(D1.east)+(0,.4)$) -- ($(V1.west)+(0,.4)$) node [midway, above] {Transform};
		\draw[thick,->] ($(D2.east)+(0,.4)$) -- ($(V2.west)+(0,.4)$) node [midway, above] {Transform};
		\draw[thick,->] ($(D3.east)+(0,.4)$) -- ($(V3.west)+(0,.4)$) node [midway, above] {Transform};

		\draw[thick,<-] ($(D1.east)-(0,.4)$) -- ($(V1.west)-(0,.4)$) node [midway, above] {Test};
		\draw[thick,<-] ($(D2.east)-(0,.4)$) -- ($(V2.west)-(0,.4)$) node [midway, above] {Test};
		\draw[thick,<-] ($(D3.east)-(0,.4)$) -- ($(V3.west)-(0,.4)$) node [midway, above] {Test};

	\end{tikzpicture}
	\caption{Diagram of the V model \citep{MDSD_INTRO}}
\label{fig:v_model}%
\end{figure}


\begin{figure}
	\centering
	\begin{tikzpicture}[
		start chain=going below,
		squarednode/.style={
			rectangle,
			draw=black,
			very thick,
			minimum height=1.5cm,
			text width=2.5cm,
			inner sep=2mm,
			align=center
		},
		every loop/.style={
			looseness=2,
			text width=2cm
		}
		]
		\node[squarednode] (D1) {System\\Design Model};
		\node[squarednode] (D2) [below = of D1, xshift = .2cm] {Architecture\\Design Model};
		\node[squarednode] (D3) [below = of D2, xshift = .2cm] {Component\\Design Model};
		\node[squarednode] (V1) [right = 3cm of D1] {System\\V\&V Model};
		\node[squarednode] (V2) [below = of V1, xshift = -.2cm]{Architecture\\V\&V Model};
		\node[squarednode] (V3) [below = of V2, xshift = -.2cm] {Component\\V\&V Model};

		\node[squarednode] (GC) at ($(D3)!0.5!(V3)$) [yshift=-3.5cm, text width=5.5cm] {Design + V\&V Artifacts\\(Source code, Glue code,\\Config. Tables, Test Cases,\\Monitors, Fault Trees, etc.)};

		\draw[thick,->] (D3.south) -- (GC.north -| D3) node [midway, left, align=right] {Code generation};
		\draw[thick,->] (V3.south) -- (GC.north -| V3) node [midway, right, align=left] {Test generation};

		\draw[thick,->] (D1) edge [loop left, align=right] node {Design rules} ();
		\draw[thick,->] (D2) edge [loop left, align=right] node {Design rules} ();
		\draw[thick,->] (D3) edge [loop left, align=right] node {Design rules} ();

		\draw[thick,->] (V1) edge [loop right, align=left] node {Formal methods} ();
		\draw[thick,->] (V2) edge [loop right, align=left] node {Formal methods} ();
		\draw[thick,->] (V3) edge [loop right, align=left] node {Formal methods} ();

		\draw[thick,->] (D1.south) -- (D2.north) node [midway, left, align=right] {Derive};
		\draw[thick,->] (D2.south) -- (D3.north) node [midway, left, align=right] {Derive};
		\draw[thick,<-] (V1.south) -- (V2.north) node [midway, right, align=left] {Use};
		\draw[thick,<-] (V2.south) -- (V3.north) node [midway, right, align=left] {Use};

		\draw[thick,<->] ($(D1.east)+(0,.4)$) -- ($(V1.west)+(0,.4)$) node [midway, above] {Transform};
		\draw[thick,<->] ($(D2.east)+(0,.4)$) -- ($(V2.west)+(0,.4)$) node [midway, above] {Transform};
		\draw[thick,<->] ($(D3.east)+(0,.4)$) -- ($(V3.west)+(0,.4)$) node [midway, above] {Transform};

		\draw[thick,<->] ($(D1.east)-(0,.4)$) -- ($(V1.west)-(0,.4)$) node [midway, above] {Test};
		\draw[thick,<->] ($(D2.east)-(0,.4)$) -- ($(V2.west)-(0,.4)$) node [midway, above] {Test};
		\draw[thick,<->] ($(D3.east)-(0,.4)$) -- ($(V3.west)-(0,.4)$) node [midway, above] {Test};

	\end{tikzpicture}
	\caption{Diagram of the Y model \citep{MDSD_INTRO}}
\label{fig:y_model}%
\end{figure}

\subsection{Developement models}

\subsubsection{V model}

The V model is a traditional approach in the software design process. The V model starts with \emph{requirements}.

\subsubsection{Y model}

The Y model (\cref{fig:y_model}) depicts a basic workflow of model based software development.

Compared to V model, Y model moves to a purely model-based solution. In the V model, the developer might use modeling tools, and implement the components by hand, with the aid of the models. On the other hand, Y model is a pure modeling process. A consequence of this approach is the transformability of the models. Many elements, like the V\&V elements can be generated from other elements.

Every design level has it's own design rules it must conform to. The detailed levels of the system must be traceable to the upper levels. The artifact of the design process is the generated code, and generated test cases.

\section{Verification techniques}
\label{sec:verification_techniques}

\subsection{Design time verification}

Design time verification is the analysis of a software in the design process. It's commonly utilized as a part of a model driven system development, where we specify the entire software --and hardware if needed-- as a model.
The models engineers produce are engineering models. The designers can understand such models easily, but they are not mathematically correct. A widely used engineering modeling standard is the \uml{}~\cite{UML}.

It is common to transform the engineering models of the system into mathematically precise constructs. After the transformation, formal verification can be used, which can analyze the behavior of systems in a mathematically precise way.

Model checking is a technique for verifying finite-state concurrent systems such as sequential circuit designs and communication protocols. It has a number of advantages over traditional approaches that are based on simulation, testing and deductive reasoning. In particular, model checking is automatic and usually quite fast. Also, if the design contains an eror, model checking will produce a counterexample that can be used to pinpoint the source of the error~\cite{mcmillan1993symbolic}.

The spreading of sensor network and various communication technologies results in an increasing amount of distributed systems. This leads to the growing demand for analysis and verification techniques.

\subsection{Runtime verification}

Design time verification, which is performed while developing the software components, runtime verification checks the systems behavior according to the system specification during operation.

Runtime verification can be used to detect errors in the system runtime. These errors might originate from:
\begin{itemize}
	\item Source code oriented failures, caused by faulty implementation. The monitoring of these faults often have hardware support, like a watchdog timer.
	\item Requirement based failures, where the system cannot satisfy the system requirements. Embedded monitors can be used to detect requirement violations.
\end{itemize}

These monitors can detect errors of the following types:
\begin{itemize}
	\item Operational errors. These errors come from the monitored application, e.g.\ implementation failures, hardware failures.
	\item Configuration errors. These errors are caused by configuration error e.g.\ syntactically incorrect configuration file, invalid IP address.
	\item Environmental errors. Environmental errors come from outside the monitored environment e.g.\ networking, or control systems.
\end{itemize}

The main purpose of runtime verification is to detect errors, but it can also utilize techniques to resolve these errors. Such monitors called \emph{reactive}. Reactive systems might be:
\begin{itemize}
	\item Operational-mode-based. Operational modes are predefined configurations of the system. These configurations can be changed in case of errors e.g.\ failsafe mode.
	\item Recovery-based. The monitor contains actions, and these actions can be fired by errors to restore the system operation.
\end{itemize}

With these techniques by monitoring the current state of the system, runtime verification can detect, and in some special cases correct the erroneous behavior of the system.

\section{Automata}
\label{sec:automata}

Automata is a mathematical formalism widespread in computer science.

An automata is a 5-tuple of $(Q, \Sigma, \delta, q_0, F)$ where the tuple elements are:
\begin{itemize}
	\item \emph{Q:} a finite set of states.
	\item \emph{$\Sigma$:} a finite set of symbols. This set is called as an alphabet for the automaton.
	\item \emph{$\delta$:} a transition function of $\delta: Q \times \Sigma \rightarrow Q$.
	\item \emph{$q_0$:} a state, element of Q, where the automata starts initially.
	\item \emph{F}: a set of states, called acceptor states.
\end{itemize}

The representation may vary, but a common illustration of an automata is depicted in \cref{fig:example_automata}.

Transitions are depicted by a directed line marked by an element from $\Sigma$. States are not necessary, but often marked by a name. In this example automata, the acceptor state is marked by a gray background.

While executing the automata, the automata processes symbols from $\Sigma$. Let $q_i$ denote the current active state. For every input symbol $a_i$, if there is a $q_i \in \delta(q_{i-1}, a_i)$ transition, the $q_i$ state (often called active state) becomes the state on the end of the transition originating from $q_{i-1}$.

If the automata reaches an acceptor state, the series of input symbols, called a word, is accepted by the automaton.

\begin{figure}[h]
	\centering
	\begin{tikzpicture} [
			auto,
			every path/.style = {
				thick,
				->
			},
		]
		\node[circle, very thick, draw=black, minimum size=8mm] (A) [] {A};
		\node[circle, very thick, draw=black, minimum size=8mm] (B) [right = 2.5cm of A] {B};
		\node[circle, very thick, draw=black, minimum size=8mm, fill=black!10] (C) [right = 3cm of B] {C};

		\draw (A) ++(-1, 1) -- (A);
		\draw (A) edge [bend left]
			node [sloped, midway, above] {a} (B);
		\draw (B) edge [bend left]
			node [sloped, midway, below] {b} (A);
		\draw (B) edge [bend left]
			node [sloped, midway, above] {c} (C);

	\end{tikzpicture}
	\caption{An automata representation}
\label{fig:example_automata}
\end{figure}

The example automata depicted in \cref{fig:example_automata} have the following tuples:
\begin{itemize}
	\item \emph{Q:} \{A, B, C, D\}
	\item \emph{$\Sigma$:} \{a, b, c\}
	\item \emph{$\delta$:} In the following list, only the transitions which cause a new state to be the active will listed
		\begin{itemize}
			\item $A \in Q \text{ and } a \in \Sigma \rightarrow B$
			\item $B \in Q \text{ and } b \in \Sigma \rightarrow A$
			\item $B \in Q \text{ and } c \in \Sigma \rightarrow C$
		\end{itemize}
	\item \emph{$q_0$:} A
	\item \emph{F}: \{C\}
\end{itemize}

The words accepted by this automata can be described with regular expression, because with symbols \texttt{a} and \texttt{b}, we could loop forever. Therefore a regular expression is needed, which can describe this. The expression which describes the accepted words is the following: \verb+ac|(ab)*c+

\subsection{Timed automata}

Timed automata is the extension of the automata. Timed automata have an additional set $C$, which contains the clocks of the automata. Transitions have clock as a guard, and transitions might reset clocks element of $C$.

\section{Complex event processing}

Complex event processing (\cep{}) is a set of techniques and tools to help us understand and control event-driven information systems \citep{Luckham:2001:PEI:515781}.

Complex event processing frameworks are practical solutions for processing large amount of events by pattern matching.

Traditionally complex event processors are complex frameworks like databases, and modeling frameworks. The aim of this thesis to generate small, but effective code, and run it on embedded environments, where it's impossible to run a full \cep{} software stack.

In a typical \cep{} use case, the developer writes patterns, consisting of events of the input or other patterns (\cref{subsec:cep_patterns}).

While its relatively easy to match simple patterns against the incoming on-line stream of information, the complexity of the problem significantly increases with:
\begin{itemize}
	\item pattern composition
	\item parallel matching of patterns
	\item temporal properties
	\item parametrization
\end{itemize}

For instance, in finance fraud detection is a common use-case for applying \cep{}. The bank receives a huge amount of payment information daily, and wants to filter out cases, where frauds may happened. The suspicious cases have the following properties:
\begin{itemize}
	\item The same person's credit card
	\item With some temporal property e.g.\ in a day
	\item Used somewhere outside from the country the last 2 payments were done
\end{itemize}

With the capability to find the occurrences of such events, complex event processing is usable to monitor the behavior of the system.

The basic workflow of a \cep{}-based monitoring framework is illustrated in \cref{fig:cep_monitoring}. If the developement is model based, according to the Y-model, from the high level specification one can generate complex event patterns (\cref{subsec:cep_patterns}). After compilation, or interpretation of the patterns, the monitor can be implemented by a \cep{} engine (\cref{subsec:cep_engine}). The generated monitor is then able to indicate the violation of the high level specification.

\begin{figure}
	\centering
	\begin{tikzpicture}[
		start chain=going below,
		node distance=5mm,
		squarednodec/.style={
			squarednode,
			on chain,
			text width = 4cm,
			align = center
		},
		]
		\node[squarednodec] (A) {High level specification};
		\node[squarednodec] (B) {Pattern creation};
		\node[squarednodec] (C) {Generation of monitor};
		\node[squarednodec] (D) {Monitor execution};

		\draw[thick, ->] (A) -- (B);
		\draw[thick, ->] (B) -- (C);
		\draw[thick, ->] (C) -- (D);

	\end{tikzpicture}
	\caption{A \cep{} based monitoring solution}
\label{fig:cep_monitoring}
\end{figure}

\noindent In the following sections, we overview the basic definitions of the CEP domain.

\subsection{Event}
\label{subsec:cep_event}

An event is an instance of an event type, which is distinguished by an unique name.
Event is the input of a complex event processor. Events can be classified into two main categories:
\begin{itemize}
	\item Atomic event. An event the indicator of a change in the environment which has to be processed. It refers to a single point of time, and it cannot be broken down into other events.
	\item Complex event. Complex events are built from other events representing a complex temporal pattern of one or more events in time.
\end{itemize}

\needspace{10em}
\subsection{Event streams}
\label{subsec:cep_eventstream}
\label{subsection:event_streams}
The input of a \cep{} engine are streams of atomic events. Atomic events represent various information from the environment:
\begin{itemize}
	\item Direct event stream. This is an imperative approach, because the events pushed programmatically from code through an \textls{API}.

	\item Model change events. The application have an underlying model and framework, and it can notify the event processor in case of various changes (deletion, insertion, update) of the model.\label{item:model}

	\item Databases. A database can be considered as a model for the data, so that events come from a database, marking the changes of it. The database may contain the model itself, or any data represented in a completely different structure.
\end{itemize}
\vspace{1ex}
If a \cep{} supports it, it is possible to mix input streams, and rules can interconnect the event sources.

\subsection{Patterns}
\label{subsec:cep_patterns}
Event processors provide formalisms to describe patterns.

The complex nature of the rules comes from the usage of the reference of other patterns. One pattern might depend on the activation of another patterns.

With this functionality, one can build hierarchical system of pattern. Instead of creating one big pattern --if it is even possible-- containing all the functionality, divided smaller, reusable patterns can be written.

\subsection{Engine}
\label{subsec:cep_engine}

The basic component of \cep{} is the execution engine. The engine accepts input streams, and the output are the matchings of the patterns. In monitoring situations the output of an engine is the event of an erroneous state of the monitored process (\cref{fig:cep_monitoring_event_flow}).

\begin{figure}
	\centering
	\begin{tikzpicture}[
		start chain=going below,
		node distance = 3mm,
		squarednodec/.style={
			squarednode,
			on chain,
			align = center,
		}
		]
		\node[squarednodec] (A) {Direct stream};
		\node[squarednodec] (B) {Model change};
		\node[squarednodec] (C) {Database};
		\node[squarednodec, minimum height=2cm] (D) [right = 2cm of B] {CEP Engine};

		\draw[thick, ->] (A.east) -- (D);
		\draw[thick, ->] (B.east) -- (D);
		\draw[thick, ->] (C.east) -- (D);
		\draw[thick, ->] (D.east) -- ++(4,0) node [above, midway] {Output/Observations};

		\draw[thick, ->] (6.5,0.5) -- ++(2,0) node [pos = 1.5] {Event flow};

	\end{tikzpicture}
	\caption{Event flow in a \cep{} based monitor}
\label{fig:cep_monitoring_event_flow}
\end{figure}

\section{Real-time systems}

Real-time systems have some temporal constraints (deadline), which the system must satisfy. In a real-time system the correctness of the behavior not only depends on the control logic, but on temporal properties as well. These constrains often focused on the response time of the system in case of events.

We can divide real-time systems into two main categories: hard and soft real-time.

\subsubsection{Hard real-time}

Hard real-time systems have precisely defined deadlines which the system must satisfy. If the system misses a deadline, it is considered a failure. An example of a hard real-time system is an airbag controller inside a car. After a collision, there is a very narrow time window where the airbag must be inflated. A hard real-time system must have some proof they can deterministically handle these situations without missing any deadlines.

\subsubsection{Soft real-time}

The definition of soft real-time differ from hard real-time system in the way they stay correct if they miss the deadline, but operation between time constraints needed for normal functionality. One example is the processing of mouse input. If the processing of mouse input lags, it can be annoying for the user. Nonetheless the functionality stays intact, but the usability seriously degrades.

\subsection{Realtime operating systems}

The type of an operating system is defined by how the scheduler decides which program to run and when to run it. For example, the scheduler used in a multi-user operating system (such as Unix) will ensure each user gets a fair amount of the processing time. As another example, the scheduler in a desktop operating system (such as Windows) will try to ensure the computer remains responsive.

The scheduler in a \emph{Real Time Operating System} (\rtos) is designed to provide a predictable\,--normally described as deterministic--\,execution pattern. This is particularly of interest to embedded systems as embedded systems often have real time requirements. A real time requirements is one that specifies that the embedded system must respond to a certain event within a strictly defined time (the deadline). A guarantee to meet real-time requirements can only be made if the behavior of the scheduler of the operating system can be predicted (and is therefore deterministic).\citep{RTOS}

Most of the safety-oriented systems are based on realtime operating systems. It is important to understand realtime operating systems exists to provide determinism, and not additional performance. With these response guarantees, the software engineer can be sure the incoming interrupt will be served, and the software components responsible for the processing will get \cpu{} time between the guaranteed limits.

\section{Mission critical systems}
\label{sec:mission_critical_system}

A system is considered to be mission-critical, if a system failure can cause severe financial loss, natural disaster, or loss of human lives. Most of the time mission critical systems utilizes hard-realtime systems. Association between mission critical, and hard-realtime is common, but there are counterexamples:

One example is a bank transaction software. A transaction execution software does not need to execute transactions in real-time, but if a transaction does not executed correctly, it can mean serious financial loss.

Another example is a printer head. A printer head is a hard real time component, because if it hits the specified deadline, the output will not be correct. A misprinted page from a desktop printer doesn't meet the definition of a mission critical system.

\section{Domain specific language}

A domain-specific language (\dsl{}) is a programming language or executable specification language that offers, through appropriate notations and abstractions, expressive power focused on, and usually restricted to, a particular problem domain \citep{van2000domain}.

\dsl{}s are useful, because the language's designer completely in control of the language, and the constraints. These constrains can help the designer get real-time feedback about the correctness of the code written in \dsl. For example an automaton definition \dsl{} can support design time checking states of an automaton, and provide feedback to the designer which states are unreachable.

\section{Eclipse}

Eclipse is an open source integrated development environment (\textls{IDE}) actively developed since 2001. Eclipse is the main platform of modeling tools implementing industry wide standards e.g. \emf, \textls{AUTOSAR}, \textls{UML}, \textls{SysML}.

Eclipse achieve this modularity with its radical usage of plug-ins. The entire \textls{IDE} is built around plug-ins, executed by a thin, runtime layer.

To make such a distributed plug-in network usable, Eclipse utilizes the Open Services Gateway initiative (\textls{OSGi}) framework. The \textls{OSGi} framework uses the extra metadata bundled with Java packages to handle dependency requirements inside the runtime, and to interconnect the plug-ins with each other.

\subsection{Eclipse Modelign Framework}

The Eclipse Modelign Framework (\emf) is a well development modeling framework to support model creation for model driven development.

Many frameworks, like the \viatra{} framework. Such frameworks can use the notification support coming out of the box with \emf{} to support advanced incremential features:
\begin{itemize}
	\item Visual model editing. The \emf{} editor gives a tree based, or completely cal editing of it's models.
	\item Notification support for change events.
	\item Code generation features.
	      \begin{itemize}
	      	\item Model code generation. The framework can generate Java classes, which can be used from any Java application.
	      	\item Editor code generation. A generated editor as an Eclipse plugin, which gives the developer a basic tool to assemble an instance model.
	      \end{itemize}
\end{itemize}

\section{Related work}

\begin{itemize}
	\item The \viatrac{} framework contains a simulator, which can work on the compiled metamodel.
	\item The work of \cite{TDK2015} present a similar solution, but the formalism behind the generation is a transition system. The monitor generated heavily utilizing \cpp{11} features, which is not available in many embedded platforms.
	
\end{itemize}
