\chapter{Programmable Realtime Unit}
\label{chap:pru}

This chapter will discuss the capabilities, and usage of the Programmable Realtime Unit (\textls{PRU}).

\section{Overview}

Even with realtime operating systems, no process can allocate hundred percent time of a CPU. There are essential OS management tasks, which handles the hardware. These management tasks needs to be executed periodically.
\\[1ex]
Communication protocols have various need. Two example are the Inter-Integrated Circuit (\isc), and the Universal Asynchronous Receiver Transmitter (\uart) protocol. Where no dedicated \isc or \uart hardware is available on a specific platform, the protocol is implemented in software. This software based protocol implementation is called \emph{bit--banging}, because the driver is toggling the \gpio pins according to protocol specification.

Let's review with the \uart bit--banging example, and what is the problem with it.

\subsection{Example}

\begin{figure}[h]
	\centering
	\begin{tikzpicture}[
			squarednode/.style={
				rectangle,
				draw=black,
				very thick,
				minimum height=2cm,
				minimum width=2.5cm
			},
			label/.style={
				font=\footnotesize
			}
		]
		\node[squarednode] (A) [] {};
		\node[squarednode] (B) [right = 3cm of A] {};

		\node (A_label) [right = 2mm of A.west] {UART$_1$};
		\node (B_label) [left = 2mm of B.east] {UART$_2$};

		\draw[thick,->] (A.east) ++(0,0.5) -- ++(3,0) node[label,pos=-0.15] {TX} node[label,pos=1.15] {RX};
		\draw[thick,->] (B.west) ++(0,-0.5) -- ++(-3,0) node[label,pos=-0.15] {TX} node[label,pos=1.15] {RX};
	\end{tikzpicture}
	\caption{A common \uart configuration}
	\label{fig:example_uart}
\end{figure}

In this example, the \uart bit--banging solution will be presented. The \uart is an asynchronous protocol, meaning it does not have a common clock signal (\cref{fig:example_uart}). In an asynchronous protocol, the two side (UART$_1$, and UART$_2$) have an internal clock. Baud is the number of symbols\,---in \uart, bits---\, can be sent in a second, and the internal clock is set to sample this rate correctly. Some de--facto baud rates used in the industry are for example: $9600, 19200, 38400, 57600, 115200$

One perfectly sampled \uart communication is showed in \cref{fig:uart_correct}. The communication starts with the sender pulling down the signal line down. This marks the start of transmission. The other side senses the falling edge, and preparing for the transmission. After the start signal, the sender will send the data with the pre defined baud rate. The receiver side is sampling the data when the transmitter side is between the transitions.

\newcounter{countup}
\setcounter{countup}{-1}
\newcommand*{\countup}{\addtocounter{countup}{1}\thecountup}

\begin{figure}[h]
	\centering
	\setcounter{countup}{-1}
	\begin{tikztimingtable}[yscale=2.0,timing/wscale=1.5]
		Signal								& 2H 2L 8{2D{D\countup}} 2H 2H \\
		Sample								& 2L {.2L .3H 1.5L} 9{.8L .4H .8L} 2L \\
	\end{tikztimingtable}
	\caption{Timing of a correct \uart communication}
	\label{fig:uart_correct}
\end{figure}

\begin{figure}[h]
	\centering
	\setcounter{countup}{-1}
	\begin{tikztimingtable}[yscale=2.0,timing/wscale=1.5]
		Signal								& 2H 2L 2.4D{D\countup} 2.8D{D\countup} 1.8D{D\countup} 2.25D{D\countup} 2D{D\countup} 1.75D{D\countup} 1.86D{D\countup} 2.1D{D\countup} 2H 2H \\
		Sample								& 2L {.2L .3H 1.5L} 9{.8L .4H .8L} 2L \\
	\end{tikztimingtable}
	\caption{Timing of an incorrect \uart communication}
	\label{fig:uart_incorrect}
\end{figure}

One weak point of the \uart communication is the need of precise, synchronized timing on both sides. This is caused by the lack of the clock signal between the two transceivers.\footnote{This problem is solved by the Universal Serial Receiver Transmitter (\usrt), but it needs another signal line, so other more capable protocols like Serial Peripherial Interface (\spi), or \isc replaced the \usrt.} \cref{fig:uart_incorrect} shows an incorrect timing on the sender side. Every error in the transmission is accumulated, and false sampling can happen. In our example, \emph{D1} data bit is sampled twice, and we completely miss the stop sign.
