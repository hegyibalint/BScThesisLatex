%!TEX root = ../main.tex

\chapter{Code Generation}
\label{ch:codegen}

This chapter will introduce the basics of code generation, and depict the process used to generate the code used in the case study.

\section{Introduction}

Code generation is the process of transformation of a domain-specific language to another domain-specific language. The transformation's input and output type can be different, the main combinations are the following:
\begin{itemize}
	\item \mtt{}: Model to text generation. This generation takes a model, and generate textual artifact from it. The generated text can be source code, documentation, or any textual artifact. An example for \mtm{} generation is the generation of JavaDoc. The input is the AST of the Java code, and the output is the textual documentation of the code.
	\item \mtc{}: Model to code generation. \mtc{} is a special case of \mtt{}, where the artifact is compilable, or executable code from a model. One use case of \mtc{} generation is the final step of a model driven software development.
	\item \mtm{}: Model to model generation. \mtm{} generators take a model, and transform it to another model.
\end{itemize}

\cref{fig:model_transformations} depict an example of these transformations, where the source and result of the generation is textual. This textual source can be any programming language like Java, \cpp{}. \cref{fig:model_transformation_cpp} depict a situation where the developer create a textual representation of a finite automaton (FA) in a given formalism (Code$_\text{FA}$).

This formalism can be converted into a model (Model$_\text{FA}$). With \mtm{} transformation, the model can be converted into a model (Model$_\text{\cpp{}}$), which is applicable of the generation of code output (Model$_\text{FA}$).

This example converts from a domain specific language of finite automation to the domain specific language of an imperative programming language.

\begin{figure}
	\centering
	\begin{tikzpicture}
		\node[squarednode] (M1) [] {Model$_1$};
		\node[squarednode] (T1) [below = of M1] {Text$_1$};

		\node[squarednode] (M2) [right = 2cm of M1] {Model$_2$};
		\node[squarednode] (T2) [below = of M2] {Text$_2$};

		\draw [thick, <->] (M1) -- (M2) node [midway, above] {M2M};
		\draw [thick, densely dashed, <->] (T1) -- (T2) node [midway, above] {T2T};
		\draw [thick, <->] (M1) -- (T1) node [midway, left] {M2T/M2C};
		\draw [thick, <->] (M2) -- (T2) node [midway, right] {M2T/M2C};
	\end{tikzpicture}
	\caption{Usage of code generation from text to text}
\label{fig:model_transformations}
\end{figure}

\begin{figure}
	\centering
	\begin{tikzpicture}
		\node[squarednode] (M1) [] {Model$_\text{FA}$};
		\node[squarednode] (T1) [below = of M1] {Code$_\text{FA}$};

		\node[squarednode] (M2) [right = 2cm of M1] {Model$_\text{\cpp{}}$};
		\node[squarednode] (T2) [below = of M2] {Code$_\text{\cpp{}}$};

		\draw [thick, ->] (M1) -- (M2) node [midway, above] {\mtm};
		\draw [thick, densely dashed, <->] (T1) -- (T2) node [midway, above] {T2T};
		\draw [thick, <-] (M1) -- (T1) node [midway, left] {\ttm};
		\draw [thick, ->] (M2) -- (T2) node [midway, right] {\mtt};
	\end{tikzpicture}
	\caption{Workflow of a real-world usage of code generation}
\label{fig:model_transformation_cpp}
\end{figure}

\section{Code generator structure}

The structure of a given code generator describes it's performance, reusability, and maintainability.

\subsection{Dedicated code generator}
\label{subsec:dedicated_code_generator}

Dedicated code generators described as generators where the generator itself are black box, and the (\cref{fig:dedicated_code_generator}). The developer creates a model, which can be processed by the generator, and converts it by the code generator. The reason behind this black-box model is because the huge intellectual property these code generators represents. For example Mathworks' Embedded Code Generator, or Simulink \citep{MATHWORKS_ECG}\citep{MATHWORKS_SIMULINK} considered as a dedicated code generator.

\begin{figure}
	\centering
	\begin{tikzpicture}[
			every node/.style = {
				align=center
			}
		]
		\node[squarednode] (M) [] {Model};
		\node[squarednode] (DCG) [right = of M, text width=3cm] {Dedicated Code Generator};
		\node[squarednode, fill=black!10] (TA) [right = of DCG] {Textual artifact};

		\draw [thick, ->] (M) -- (DCG);
		\draw [thick, ->] (DCG) -- (TA);
		\draw [thick, double equal sign distance, <-] (DCG.north) -- +(0,1) node [pos=0.65, right, xshift=1mm] {Parameters};
	\end{tikzpicture}
	\caption{Dedicated code generator structure}
\label{fig:dedicated_code_generator}
\end{figure}

\subsection{Template based generator}
\label{subsec:template_based_generator}

Template based code generators (\cref{fig:template_code_generator}) offer a different approach other than dedicated code generators (\cref{subsec:dedicated_code_generator}). Instead of model only input, which serves as data source for the generator, template based generators follows a more flexible approach.

Next to the model, the designer provides a template. This template combined with the data generates a valid output.

\begin{figure}
	\centering
	\begin{tikzpicture}[
			every node/.style = {
				align=center,
				anchor=north west
			}
		]
		\node[squarednode] (M) at (0,0) [] {Model};
		\node[squarednode] (T) at (0,1) [below = of M] {Template};
		\node[squarednode] (DCG) at (4,0) [text width=4cm, minimum height=3cm] {}
		node [below = 0.1cm of DCG.north] {Template\\compiler/generator}
		node [squarednode] [above = 0.25cm of DCG.south, text width=3cm] {Executable\\template code};
		\node[squarednode, fill=black!10] (TA) [right = of DCG] {Textual artifact};

		\draw [thick, ->] (M.east) -- ++(1.05,0);
		\draw [thick, ->] (T.east) -- ++(1.05,0);
		\draw [thick, ->] (DCG) -- ++(TA);
		\draw [thick, double equal sign distance, <-] (DCG.north) -- +(0,1) node [pos=0.65, right, xshift=1mm] {Parameters};

	\end{tikzpicture}
	\caption{Template code generator structure}
\label{fig:template_code_generator}
\end{figure}

One of the proven template based generator is the Xtend language. Before getting into Xtend, the following sections will represent the technologies Xtend is built around.

\section{Tools}

\subsection{ANTLR}

\textls{ANTLR} (ANother Tool for Language Recognition) is a powerful parser generator for reading, processing, executing, or translating structured text or binary files. It's widely used to build languages, tools, and frameworks. From a grammar, ANTLR generates a parser that can build and walk parse trees \citep{ANTLR}.

\textls{ANTLR} is a very powerful framework, not just because it's expressiveness, but the support of runtime languages it can generate parsers to. \textls{ANTLR} supported runtime languages are:
\begin{itemize}
	\item \textls{C\#}
	\item Java
	\item JavaScript
	\item Python2
	\item Python3
\end{itemize}

Besides the runtime languages, various other languages are implemented in form of grammars, so they became parsable by \textls{ANTLR}. Currently 88 various languages have \textls{ANTLR} grammar.

\subsection{Xtext}

Xtext is a frameworks built around a subset of \textls{ANTLR}. The reason behind the restriction of \textls{ANTLR} language's expressiveness  is to make it compatible with Eclipse technologies like \emf{}, which is the backbone of Xtext. The language written in Xtext automatically translates to an \emf{} metamodel, which serves as an abstract syntax tree (\textls{AST}) metamodel. As the developer writes a DSL previously defined by an Xtext language, the corresponding \emf{} metamodel is instantiated

Xtext, like EMF integrates itself into Eclipse. Xtext languages generates editors with code completion support, and with other technologies like Sirius, graphical editing features can be implemented without coding.

\subsection{Xtend}

Xtend is an Xtext languange. Xtend is a language specialized in \mtt{} code generation, but can be used as a mixed functional-imperative object-oriented programming language. The following sections will depict the functionalities useful for \mtt{} generation.

Xtend is compiled to Java source code, therefore the code can be reviewed with tools specialized to Java.

\subsubsection{Extension methods}

One of the most powerful extension to the Java languages in Xtend are extension methods. The basic concept comes from the \textls{C\#} language, where types can be extended with methods next to existing types methods implemented inside the type's source code.

While \textls{C\#} uses a special syntax to mark the extension methods (the first parameter marks the extended object, and the parameter is marked with the \emph{this} keyword), Xtend identify extension methods by the first parameter of the method, or using another variable's methods as extension methods with an additional \emph{extension} keyword in the variable declaration.

\subsubsection{Template expressions}

From the code generating point of view, template expressions are the most useful feature of the language. With template expressions, one can write strings which serves as the generation templates. Template expressions defined by triple single quotes (\verb+'''+). Inside the template the guillemets (\verb+«»+) act as an escape character. Inside these guillemets one can write Xtend code, e.g.\ access iterator, or local variables.

An example of template expression usage presented in \cref{lst:template_expression_example}. With the usage of template expressions, information of a model can be injected into the template, according to the description of \cref{subsec:template_based_generator}.

\begin{lstlisting}[
	float,
	caption={Sample template expression in Xtend},
	label=lst:template_expression_example,
	language={Xtend}
]
var acceptCode = '''
	void Automaton_[gl]a.name[gr]::accept() {
		// Implement accepting action here
		return;
	}
'''
\end{lstlisting}

\subsubsection{Template keywords}

The template syntax is expanded with keywords including:
\begin{itemize}
	\item FOR, and ENDFOR
	\item IF, and ENDIF
\end{itemize}
which can be used inside guillemets. These syntaxes helps the generation of lists, and special conditionally generated source blocks.

\subsection{\viatraq/\incq}

\viatraq{} \citep{ujhelyi2015emf} is an incremental graph pattern matcher over the \emf{} framework. In the year 2016, the project merged with the \viatra{} framework, and \incq{} was renamed to \viatraq.

Code generation, model transformation, checking well formed constraints are problems which requires model traversal. One traditional technique of model traversal is the visitor pattern \citep{gamma1995design}. While the visitor patterns is a good approach, but it lacks scalability and support of incrementality.

\subsubsection{Graph query}

Imperative approaches might be feasible in smaller models and simple model structure, but the design process of a model traversal is a repetitive task resulting in a boilerplate code with high cyclomatic complexity.

\viatraq{} solved this problem by utilizing \rete{} networks. The approach of \rete{} networks and how the required information if selected is very alike of relational calculus. One big added feature of the language is the support of transitive closure.

The \dsl{} built for \viatraq{} is an Xtext based language with strong code-completion features deeply integrated into Eclipse and \emf{}. This \dsl{} supports writing queries in a declarative way, resulting in more effective design workflow and reduced debugging times.

\begin{lstlisting}[
	float,
	caption={Sample Viatra query pattern},
	label=lst:viatraq_example_pattern,
	language={ViatraQ}
]
pattern EpsilonTransition(a: Automaton) {
	Automaton.states.incomingTransitions(a, t);
	EpsilonTransition(t);
} or {
	Automaton.states.outgoingTransitions(a, t);
	EpsilonTransition(t);
}
\end{lstlisting}

\subsubsection{Runtime}

While the design process workflow is coupled to Eclipse, the \viatraq{} runtime can be used in plain Java applications, therefore the platforms are not restricted to Eclipse installations.

\section{Generation of monitors}

This section describe the code generator implemented for the metamodel described in \cref{sec:metamodel}.

\subsection{Supported features}

The metamodel support automaton formalism with temporal, and parametrized behavior. An important phase of development is the definition of the supported subset of the automata formalism, which can be generated.

\subsubsection{Basic automaton}

A basic automata mentioned here is the one defined in \cref{sec:automata}. Generating automaton from such formalism can be done in a way the generated code have very static, deterministic size. The only memory the executions needs is a value, which stores the active state of the automata. From this information, the automaton can be stepped by generating control structures.

\subsubsection{Timed automaton}

Timing can be achieved in a static way, like the basic automaton. States have clocks, and there is an event called \emph{symbolicTimeoutEvent}. To expand the basic automaton with timing, the automata must store the clock values. Notice, the number of clocks is determined in the compile process, meaning there is no dynamic allocation. All the memory the timed automata needs can be statically allocated.

\subsubsection{Parametrized automaton}

With parametrized automaton, one can't stay completely static. With every new parameter, some dynamic allocation should be done. In the worst case, these dynamically allocated regions never freed. With limited resources, there is a little room of error, and the data memory can be filled quickly with abandoned tokens.

\subsubsection{Conclusion}

Because of the complex behavior of the parametrized automaton, and the inevitable use of dynamically allocation with it, the code generator will use timed automaton as a target formalism. With this formalism, the memory need is known while compiling the automata.

\subsection{Applied technologies}

As mentioned before, generic simulation is one tool to execute automaton, but it's resource needs might exceed what is the hardware capable of. In this case, \mtc{} code generation is a good tool, which can be used to generate specific, efficient code.

To maximize the flexibility of the generator, the code generation uses Xtend, as a generator language. With Xtend, and related technologies like \viatraq{}, a template-based code generator can be created. While Xtend can be used to generate code with template expressions, the language itself became a general purpose language, and instead of Java, the entire project is developed in Xtend.

The code generator is implemented as Eclipse plugins. This decision is based on the fact the \viatrac{} project uses Eclipse as a platform. With the correct structure, this code generator can contribute to this. This thesis will not cover the integration process, but it's definitely a future goal.

\subsection{Project structure}

The main project structure is the following:
\begin{itemize}
	\item \emph{hu.bme.mit.inf.cepgen.common:} While the current focus is on \cpl/\cpp{} language generation, a future goal is to support other languages, and platforms. To help further development, this plug-in provides abstract classes, and interfaces. With this abstraction level, further platforms and languages can be added with low implementation cost.
	\item \emph{hu.bme.mit.inf.cepgen.query:} This project contains all the \viatraq{} queries. These queries are placed in a separate plug-in, to provide reusability. These patterns provide generic, automaton related queries.
	\item \emph{hu.bme.mit.inf.cepgen.pru:} This project is a language specific (\cpl{}) generator for the \pru{} of the BeagleBone Black (\cref{ch:pru}).
\end{itemize}

\subsection{Generation workflow}

The monitors are generated from an in-memory instance model of the automaton described in \cref{sec:metamodel}. This instance model is validated against current compiler constraints. The only constraint currently the model validated against is the existence of epsilon transitions inside the automaton. If an epsilon transition is present in the automata, it means the automaton is non-deterministic. Because the code generator can only generate code from deterministic automaton. The violation of constrains result in an exception.

After validation, the symbolic states generated into a \cpp{} enumeration. The reason why the symbolic events are not contained in the automaton objects is because of the complex event processing nature. Defined events in the regular expression language describing the patterns are defined globally. As a consequence, the automaton share these event types between themselves. To make this share type safe, the symbolic events stored in a higher level of the hierarchy. The template used to generate this enumeration can be found in \cref{lst:event_enumeration_template}.

As mentioned, with the template expression features of the Xtend language, generator templates are quite readable, and with the \texttt{SEPARATOR} attribute in the \texttt{FOR} control structure, it's very easy to add separation between items, instead of an error prone, boilerplate solution.

\begin{lstlisting}[
	float,
	caption={Template used to generate the event enumeration},
	label=lst:event_enumeration_template,
	language={Xtend}
]
#pragma once

typedef enum {
	«FOR sie : cep.symbolicEvents SEPARATOR ','»
		EVENT_«sie.name»
	«ENDFOR»
} Event;
\end{lstlisting}

\subsection{Automata generation process}

\needspace{5ex}
After generating the event enumerations, each automata can be processed separately. One very key point of the \cpp{} code generation is static behavior. Because the \pru{} is so limited in resources, the executor can not have dynamic allocations. The automaton itself can be executed quite statically. An efficient control structure for the execution of an automata is a switch-case control structure. The execution control structure have the following handling mechanism:
\begin{enumerate}
	\item A switch-case which selects the corresponding event case.
	\item An embedded switch-case inside each event cases. This embedded control structure selects the corresponding active state. If a case exists, which matches the current active state, a transition can be fired. Before jumping to the next state, there might be other side effects of the transition:
	\begin{itemize}
		\item A transition might reset, or set a timer.
		\item The state which the transition leads might be an acceptor state, meaning we must specially handle this transition by calling the \emph{accept} method of the automata.
	\end{itemize}
	\item Set the active state to the state on the other end of the transition.
\end{enumerate}

By using \viatraq{} patterns, relations between states can be easily queried from a model. Queries are useful when a relation is needed, and is not present in the metamodel in a form of references. An example is \cref{lst:viatraq_example}. This query can collect out transitions, which is guarded by a specific symbolic event. The method which generates the simple automata is listed in \cref{lst:automata_handle_code}. Note this code have temporal sections, which will set, or reset clocks.

\begin{lstlisting}[
	float,
	caption={An example query of selection by symbolic event},
	label=lst:viatraq_example,
	language={ViatraQ}
]
pattern InputStateTransition(a: Automaton, sie : SymbolicInputEvent, t: Transition) {
	Automaton(a);
	Automaton.states(a, s);

	State(s);
	State.outgoingTransitions(s, t);

	Transition(t);
	Transition.eventguard.type(t, sie);
}
\end{lstlisting}

\begin{lstlisting}[
	float,
	caption={Partial code of the automata generator template},
	label=lst:automata_handle_code,
	language={Xtend},
	basicstyle=\ttfamily\small
]
void Automaton_«a.name»::handleEvent(Event inputEvent) {
	switch (inputEvent) {
		«FOR sie : cep.symbolicEvents»
			case (EVENT_«sie.name»):
				switch (this->activeState) {
					«FOR t : InputStateTransitionMatcher.on(qe).getAllValuesOft(a, sie).sortBy[t | t.from.id]»
						case «t.from.id»:
							«FOR setAction : SetTimerActionMatcher.on(qe).getAllValuesOfsta(t)»
								setTimer(«timer2index.get(setAction.timer)», «setAction.toValue»);
							«ENDFOR»
							«FOR setAction : ResetTimerActionMatcher.on(qe).getAllValuesOfrta(t)»
								resetTimer(«timer2index.get(setAction.timer)»);
							«ENDFOR»
							setActiveState(«t.to.id»);
							«IF t.to.acceptor»
							accept();
							«ENDIF»
							return;
					«ENDFOR»
				}
				return;
		«ENDFOR»
	}
}
\end{lstlisting}

With the collection of transitions having same symbolic events, the executable \cpp{} code can be generated. In the first versions of the code generator, the template expressions called other template expressions, but this resulted in a non-sequential, hardly readable code.

After this conclusion, the methods were moved into the main template code, resulting in a monolithic template expression, which resulted in a far more readable code, and more understandable program flow.

\subsection{Timer generation process}

The instance model contains timing information, and this can be used to realize the timers of the automaton.

With timed automata, the code generator must handle timers, and the timeout events. Because the timeout is an event, it would be obvious to generate the timeout transitions inside the \emph{handleEvent} method, but there is a problem: only the \emph{SymbolicInputEvent} class have a name as an unique identifier, the \emph{SymbolicTimeoutEvent} does not.

To overcome this problem, a map is created, which generate an unique identifier to the clock. Because \emph{SymbolicTimeoutEvent} never handled by the user, the automaton handling divided into two methods:
\begin{itemize}
	\item \emph{handleEvent}, which have an input parameter, and step the automata according to this event.
	\item \emph{handleTimers}, which decrease clocks, and if a timeout happens, it step the automata according to the transition with \emph{SymbolicTimeoutEvent} as a guard.
\end{itemize}

\begin{lstlisting}[
	float,
	caption={Partial code of the timer handle code},
	label=lst:automata_timer_handle_code,
	language={Xtend},
	basicstyle=\ttfamily\small
]
void Automata_«a.name»::handleTimers() {
	«FOR timer : a.timers»
		if (this->timers[«timer2index.get(timer)»].active) {
			this->timers[«timer2index.get(timer)»].t--;
			if (this->timers[«timer2index.get(timer)»].t == 0) {
				switch(this->activeState) {
					«FOR t : TimerResetTransitionMatcher.on(qe).getAllValuesOft(timer)»
						case «t.from.id»:
							setActiveState(«t.to.id»);
							resetTimer(«timer2index.get(timer)»);
							break;
					«ENDFOR»
				}
			}
		}
	«ENDFOR»
}
\end{lstlisting}


This division between the two method have a benefit of decrease in execution time. The \emph{handleTimers} most of the cases is much smaller than \emph{handleEvent}, because many transitions might originate from a state, but a state with a clock can only have one \emph{SymbolicTimeoutEvent}, meaning the control structure defining the transitions are much smaller. The code generating the handler method of timers is listed in \cref{lst:automata_timer_handle_code}
