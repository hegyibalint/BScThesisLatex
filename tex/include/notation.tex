
%
% General
%

\let\Pr\relax
\DeclareMathOperator{\Pr}{\mathbb{P}}
\DeclareMathOperator{\Ex}{\mathbb{E}}
\DeclareMathOperator{\diag}{diag}
\DeclareMathOperator{\Span}{span}
\DeclareMathOperator{\grade}{grade}
\DeclareMathOperator*{\argmin}{arg\,min}
\DeclareMathOperator*{\argmax}{arg\,max}

%
% Abbreviations
%
\newcommand*{\osx}{\textls{OSX}\xspace}

\newcommand*{\emf}{\textls{EMF}\xspace}
\newcommand*{\rete}{\textls{RETE}\xspace}
\newcommand*{\viatra}{\textls{VIATRA}\xspace}
\newcommand*{\incq}{\textls{\emf-Query}\xspace}
\newcommand*{\viatraq}{\textls{VIATRA-Query}\xspace}
\newcommand*{\dsl}{\textls{DSL}\xspace}
\newcommand*{\cpl}[1]{\textls{C#1}}
\newcommand*{\cpp}[1]{\textls{C\texttt{++}#1}}


\newcommand*{\elf}{\textls{ELF}\xspace}
\newcommand*{\ti}{\textls{TI}\xspace}
\newcommand*{\cpu}{\textls{CPU}\xspace}
\newcommand*{\pru}{\textls{PRU}\xspace}
\newcommand*{\pruss}{\textls{PRU--ICSS}\xspace}
\newcommand*{\isc}{\textls{I$^2$C}\xspace}
\newcommand*{\cu}{\textls{CU}\xspace}
\newcommand*{\gpio}{\textls{GPIO}\xspace}
\newcommand*{\cep}{\textls{CEP}\xspace}
\newcommand*{\iot}{\textls{IoT}\xspace}
\newcommand*{\pwm}{\textls{PWM}\xspace}
\newcommand*{\spi}{\textls{SPI}\xspace}
\newcommand*{\rtos}{\textls{RTOS}\xspace}

\newcommand*{\mtt}{\textls{M2T}\xspace}
\newcommand*{\mtc}{\textls{M2C}\xspace}
\newcommand*{\ttm}{\textls{T2M}\xspace}
\newcommand*{\ctm}{\textls{C2M}\xspace}
\newcommand*{\mtm}{\textls{M2M}\xspace}
\newcommand*{\abstst}{\textls{AST}\xspace}

%
%
% Figures
%

\tikzset{
  tdk highlight/.style={
    fill=white,
    drop shadow={color=black,opacity=0.1,shadow xshift=2pt, shadow yshift=-2pt}
  }
}

%
% Numbers
%

\newcommand*{\RR}{\mathbb{R}} % Reals
\newcommand*{\RRpos}{\RR^{+}} % Positiver reals
\newcommand*{\QQ}{\mathbb{Q}} % Rationals
\newcommand*{\ZZ}{\mathbb{Z}} % Whole numbers
\newcommand*{\NN}{\mathbb{N}} % Naturals
\newcommand*{\NNpos}{\NN^{+}} % Positive whole numbers

%
% UML
%

\tikzset{
  uml class/.style={draw, tdk highlight, inner xsep=5pt,font=\strut\ttfamily},
  uml not a class/.style={font=\strut},
  uml lollipop/.style={draw,circle,tdk highlight,inner sep=3pt},
  uml inheritance/.style={draw,{Triangle[open,fill=white,angle=90:8pt]}-{}}
}

\newcommand{\umlLollipop}[3][4em,2em]{
  \path (#2) edge ++(#1);
  \node at ($(#2)+(#1)$) [uml lollipop,label=0:{#3}] {};
}
